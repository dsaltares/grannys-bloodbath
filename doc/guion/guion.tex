\documentclass[titlepage]{article}

\usepackage[utf8]{inputenc} % Codificación
\usepackage[spanish]{babel} % Español
\usepackage{graphicx} % Para incluir imágenes

\author{Manuel de la Calle Brihuega\\Jose Marente Florín\\David Saltares Márquez}
\title{Guión de Granny's Bloodbath}

\begin{document}

\maketitle

\clearpage

\tableofcontents

\clearpage

\section{Introducción}

Al comenzar la aventura desde el principio aparece la escena de historia introductoria. Un fondo oscuro con una imagen distorsionada por los bordes a la izquierda en la que se puede ver a la abuela sentada en su sofá viendo la televisión. A la derecha unas letras con el siguiente texto:\\

\emph{La abuelita vivía tranquilamente en una casita de un barrio residencial. Era una zona apacible, tranquila, por eso la eligió para disfrutar de su merecida jubilación. De pronto, su descanso se ve interrumpido por... ¡Una invasión Zombie! Están por todas partes, no hay tiempo, debe coger la escopeta de caza de su difunto marido, salir en bata y luchar por su vida. ¡¡Arrastrará a todos los que pueda con ella al infierno!!}\\

El narrador lo lee con una voz distorsionada, más grave.

\section{La casa de la abuelita}

El primer nivel del juego, la abuela debe atravesar su casa y salir al exterior para salvar su vida. (Ver documento de requisitos del juego para más detalles)

\section{Logra salir a la calle}

Fragmento de historia correspondiente al momento en el que la abuela logra salir de su casa y alcanza el barrio residencial. Tenemos un fondo oscuro con una imagen con el borde distorsionado a la izquierda. En ella vemos a la abuela en la puerta de su casa. A la derecha el texto: \\

\emph{Cuando la abuelita sale al exterior lo único que ve es destrucción. El barrio apacible en el que una vez vivió había cambiado para siempre y aquellas abominables criaturas debían pagar por ello. Debía dirigirse a la comisaría de policía a pedir ayuda.}\\

El narrador lo lee con una voz distorsionada, más grave.

\section{El barrio residencial}

Llegamos al segundo nivel del juego, el barrio residencial.(Ver documento de requisitos del juego para más detalles)

\end{document}