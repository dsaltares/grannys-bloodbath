Descargar e instalar \emph{Granny's Bloodbath} es de lo más sencillo, a continuación explicamos el proceso.

\begin{itemize}
	\item Entra en la sección de descargas del blog oficial:\\
		\href{http://grannysbloodbath.wordpress.com/descargas/}{http://grannysbloodbath.wordpress.com/descargas/}
	\item Elije la última versión disponible acorde con tu sistema operativo y descárgala
	\item Según tu sistema deberás realizar uno de los siguientes pasos:
\end{itemize}

\subsection{Instalación en Windows}
Descomprime el fichero obtenido en el directorio deseado y... ¡Listo, ya puedes jugar!\\

Puede que en un futuro creemos un setup que permita elegir directorio de instalación así como creación de accesos directos. No obstante, por el momento creemos que así es suficiente. Cuanto más sencillo mejor, ¿no?

\subsection{Instalación en GNU/Linux}
Descomprime el fichero obtenido en el directorio deseado.\\

Debes tener instalados los paquetes: libsdl, libsdl-mixer, libsdl-image, libsdl-ttf. Si no los tienes abre una terminal y escribe:

\lstinputlisting[style=consola]{paquetes.sh}

Es probable que no puedas ejecutar el juego por problemas con los permisos, si ese es tu caso debes de hacer lo siguiente en la terminal:

\lstinputlisting[style=consola]{permisos.sh}

Vale, ha sido un proceso un poco más largo pero... ¡Ya puedes jugar! Puede que más adelante creemos un paquete \emph{.deb} para que la instalación sea mucho más sencilla.
