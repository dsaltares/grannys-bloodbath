Lo primero que debemos hacer es instalar las dependencias del kit de
desarrollo de PSP. Para ello abriremos una terminal e introduciremos
el siguiente comando:

\lstinputlisting[style = consola]{dependencias.sh}

Ahora debemos establecer algunas variables de entorno para que el sistema
sepa dónde encontrar las nuevas librerías de PSP a la hora de compilar.
Editamos el fichero \emph{~/.bashrc} y añadimos al final las siguientes
líneas:

\lstinputlisting[style = consola]{bash.sh}

Cuando reiniciemos el fichero \emph{~/.bashrc} volverá a cargarse pero no
es necesario hacerlo, podemos ejecutar el siguiente comando:

\lstinputlisting[style = consola]{recarga.sh}

El siguiente paso es descargarnos una copia del directorio \emph{trunk}
del repositorio de \href{http://ps2dev.org}{ps2dev}, el cual contiene todo
lo que necesitamos (y más). El repositorio tiene un tamaño considerable
y, dependiendo de cómo ande el servidor, puede tardar bastante.

\lstinputlisting[style = consola]{ps2dev.sh}

Bueno, si habéis tenido la paciencia suficiente de llegar hasta aquí vamos
por buen camino. Ahora toca instalar el \emph{toolchain}, el kit básico:

\lstinputlisting[style = consola]{toolchain.sh}

Existe un pack de bibliotecas adicionales entre las que se encuentran las
SDL llamado \emph{psplibraries}. Este pack contiene: bzip2, freetype, jpeg,
libbulletml, libmad, libmikmod, libogg, libpng, libpspvram, libTremor,
libvorbis, lua, pspgl, pspirkeyb, SDL, SDL\_gfx, SDL\_image, SDL\_mixer,
SDL\_ttf, smpeg-psp, sqlite, zlib y zziplib. Muchas son dependencias de las
SDL pero algunas como sqlite (bases de datos), lua (lenguaje de scripting) 
o pspgl (versión de Open GL para PSP) no tienen nada que ver aunque son muy
interesantes también. Lo instalamos de la siguiente manera:
 
\lstinputlisting[style = consola]{psplibraries.sh}

En teoría ya deberíamos estar listos para crear nuestros proyectos en C++
que usen las SDL para PSP, ¡pero no es así! Debe haber algún error en el
script anterior porque SDL\_mixer no se instala como debería. Hemos de
compilar e instalar sus dependencias manualmente. Comenzamos cambiando
el propietario de la carpeta donde se instala el kit de desarrollo, sino
las librerias no pueden instalarse (al menos yo no he conseguido hacerlo):

\lstinputlisting[style = consola]{permisos.sh}

Donde \emph{group} y \emph{username} son los nombres de nuestro grupo y 
usuario en el sistema.\\

Nos dirigimos a instalar libTremor manualmente, dependencia de SDL\_mixer:

\lstinputlisting[style = consola]{libtremor.sh}

Finalmente le toca el turno a SDL\_mixer y toca seguir el siguiente proceso:

\lstinputlisting[style = consola]{sdlmixer.sh}

Con eso debería bastar, siento el rodeo, si alguien encuentra una manera
mejor de instalar el kit de desarrollo para PSP con SDL en GNU/Linux, por
favor, que me lo comunique. Entre todos podemos hacer una gran guía.
